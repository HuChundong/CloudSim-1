% %This is a very basic article template.
% %There is just one section and two subsections.
\documentclass[10pt, conference, compsocconf]{IEEEtran}
\usepackage{algorithm2e}
\hyphenation{op-tical net-works semi-conduc-tor}


\begin{document}

\setcounter{page}{1}
\pagestyle{headings}
\pagenumbering {roman}

\title{A Simulated Annealing Algorithm for Energy Efficient Virtual Machine Placement }


% author names and affiliations use a multiple column layout for up to two
% different affiliations

\author{\IEEEauthorblockN{ Yongqaiang Wu, Maoling Tang}
\IEEEauthorblockA{Faculty of Science and Engineering\\
Queensland University of Technology\\
Brisbane, Australia\\
Email: \{yongqiang.wu\}@student.qut.edu.au\\
\{maolin.tang\}@qut.edu.au} \and
\IEEEauthorblockN{Warren Fraser}
\IEEEauthorblockA{Information Infrastructure Service\\
Queensland University of Technology\\
Brisbane, Australia\\
Email: \{warren.fraser\}@qut.edu.au} }


% make the title area
\maketitle


\begin{abstract}
Virtual machine placement algorithms are the key issue for the energy-efficient resource management in the data center environment. The placement problem is often modeled as bin packing problem. Due to the NP hard nature of the problem, there are only heuristic solutions, among which First Fit and Best Fit algorithms have been used widely and have generally good results. However, their simplicity leaves room to be further improved. In this paper we propose an algorithm based on simulated the annealing theory, which enables continuous improvement from any feasible state. Experiment results show that this algorithm can generate better results, saving 0-25 percentage more energy than FFD in an acceptable time amount.

\end{abstract}

\begin{IEEEkeywords}
simulated annealing; virtual machine placement; energy optimization;

\end{IEEEkeywords}

\section{Introduction}
% no \IEEEPARstart
Energy saving in data centers has become an increasingly urgent problem when
they have been growing larger and larger, consuming tremendous amount of
electricity. It is quoted in [1] that the capacity demand of data centers has
been increasing by 12 percentages annually with the electricity prices
increasing by 4.4 percentages per year, and the electricity cost in data centers
in USA is estimated at \$3.3 billion in 2008. Virtualization technology like
VmWare[2] and Xen[3], which consolidate multiple logic servers into smaller
number of Physical Machines has been used to make better use of hardware and
save energy. Instead of having its own dedicated physical host, each logic
server runs on the Virtual Machine (VM) hosted on the Physical Machine (PM).
Besides the static consolidation, live VM migration technique has also been
applied to further reduce energy consumption by migrating VMs to fewer physical
nodes when workloads are lighter.

In both static and dynamic assignment of the VMs to the PMs, the algorithms of
assigning the VMs to the PMs are the key issue, which may affect the optimized
objective like energy conservation significantly by different placement
decisions. Ideally, there is a complete solution, which provides the best result
of the optimized objective. However, this is NP hard problem which has $M^{N}$
ways of placement, where N is the physical node number and M the virtual machine
number. Therefore, heuristic solutions are usually used to solve this type of
problem, as shown in the literature [4-8].

 First Fit Decreasing (FFD);, Best Fit (BF); for Bin Packing problems are the
 well known types of them. In FFD algorithm, the balls are sorted by size in the
 descending order - largest first. The balls are then packed in the first
 available bin that can accommodate them. The bins are also ordered based on
 their efficiency, with the most efficient one in the head of the list. They
 have low complexity, good scalability and generally good results in
 practice[9]. However, their main drawback is the over simplicity that overlooks
 the large amount of combinatory possibility that may have significantly better
 result. Furthermore, it only sorts the bins in one dimension; this will reduce
 their performance when the characteristics of other dimensions disrupt the
 ordering and undermine the fundamental optimization logic.
 
Best Fit has the poor performance in providing an overall placement, because it
only aims to find the best placement for the current VM. This local optimization
strategy makes it perform poorly in the global VM assignment in the clusters.
However it has been used often in dynamic placement when only one or several VMs
need to be adjusted like in the works [10, 11].

Genetic algorithms have the potential to take into account several dimensions in
its fitness function, and are believed to be able to generate a promising result
after a certain number of generations, but there have not been very successful
attempts in tackling the VM machine placement problem so far to our best
knowledge.

In this paper, we propose the Simulated Annealing Virtual Machine Placement
algorithm (SAVMP); based on simulated annealing theory to solve this problem.
Simulated Annealing (SA); was proposed by Kirkpatrick et al to be a general
framework of heuristic solutions to NP complete optimization problems [12]. The
theory is based on the insightful analysis of connection between statistical
mechanics and multivariate or combinatorial optimization. The effectiveness of
SA comes from its extension of two basic heuristic techniques: 1) divide and
conquer approach; 2) iterative improvement scheme. It separates large changes
and small changes by allowing large changes in the objective function at high
temperatures while deferring small changes to low temperatures. It iteratively
moves the configuration by small steps from one state to another while
preventing from being stuck by allowing large changes at high temperatures.

To evaluate the performance of SAVMPA, we compare the test results of it with
those of First Fit Decreasing. Experiment results show SAVMPA has the better
performance at the cost of more time than FFD, outperforming FFD with 0-25
percentage more energy saving.

\section{Related Work}

Many researchers have proposed solutions to increase the energy efficiency of
the physical machines based on virtual machine technologies in data centers
[4-8]. Most of them make use of the ordering algorithm to generate a VM
placement. However, due to the live VM migration cost is not trivial, the FFD
cannot be used to remap all the VMs to PMs every interval. For example,
`pMapper' uses FFD to generate an initial assignment of VMs and later re-assign
VMs picked from PMs which violate resource constraints using Best Fit [4]; in
MFR algorithm, the prediction techniques are applied to deal with variable
demand proactively and FFD is applied according to the predicted demand [13];
Sandpiper introduces `grey box', `black box' technique to solve the hot spot in
clusters [6]. Our work differs from theirs in that they use the FFD as the basic
algorithm to come up solutions to meet resource requirement of the dynamic VM
workloads while conserving energy, but we are proposing an algorithm potential
to replace FFD or combine with FFD to generate a better VM placement.

Mishira et al point out the `anomalies' in some existing VM packing algorithms,
after observing some abnormal placement results by the algorithms[14]. They
propose a vector based approach in order to make use of different resources of
PMs in a more proportional way. In comparison, our proposed approach aims at
optimization of the objective function rather than achieving the balanced usage
of different resources, although the results found by our methodology should be
able to use the resources in a reasonable way.

Besides the ordering placement algorithms, Constraint Programming (CP) , a
software technology for solving combinatorial problems especially in areas of
planning and scheduling [15], is also used to allocate resources on PMs to VMs.
Entropy, for example, formulates the VM resource allocation problem firstly as a
constraint satisfaction problem then solve it by a constraint solver to get the
optimization solution. In the first step of Entropy, it finds the minimum number
of PMs, and finds a `best' migration configuration to the new state. Although CP
is a complete method, it is used here in a heuristic way because it probably
cannot finish searching in the limited time. Entropy has a certain similarity to
our proposed approach in that it can stop the searching at any time and use the
best result achieved so far.

Genetic Algorithm has been used widely as an effective heuristic way of solving
NP hard problem. It has also been applied in the VM placement problem [15, 16].
However, in the published literature, there has not been adequate proof that it
performs very well in the resource allocation problem. Campegiani et al only
used a small number of VMs and PMs in evaluation of their GA based solution ,
and Nakada et al did not use data to explain how well their algorithm performed
in their test environment. Therefore we cannot compare our proposed approach
with GA due to the lack of detail data about the GA application in the VM
placement problem.

Finally, SA is mentioned in Hser et al's work in which they designed a modified
Simulation Annealing solver to allocate VMs among PMs, but they did not provide
detail information about their SA based methodology [17].

\section{Problem Formulation}
To formulate this problem we are dealing with in the mathematical form, let's
define\\
$N$ number of virtual machines\\
$M$	number of physical machines\\
$v_i$	the number $i$ virtual machine\\
$p_j$	the number $j$ phsical machine\\
$vp_{ij}$	the binary value representing whether virtual machine $v_i$ is
assigned to physical node $p_j$\\
$V$	the set of $N$ virtual machines, namely ${v_1,v_2,...,v_N}$\\
$P$	the set of M physical machines, namely ${p_1,p_2,...,p_M}$\\
$u_j$	the percentage of CPU utilization of $p_j$\\
$e_j$	the energy consumption of $p_j$\\
$e_{max}^j$	the energy consumption of $p_j$ when $u_j$=100\%\\
$e_{idle}^j$	the energy consumption of $p_j$ when $u_j$=0\%\\
$v_{cpu}^i$	the CPU demand of $v_i$\\
$v_{mem}^i$ the RAM demand of $v_i$\\
$v_{net}^i$	the Network Bandwidth capacity of $v_i$\\
\\
$p_{cpu}^j$	 the CPU capacity of  $p_j$\\
$p_{mem}^j$ the RAM capacity of  $p_j$\\
$p_{net}^i$	the Network Bandwidth capacity of $p_j$\\
$VP$	the assignment of $V$ to $P$, represented by


$\left(
  \begin{array}{ccc}
    vp_{11} & \ldots & vp_{1M} \\
    \cdots & \ddots & \cdots \\
    vp_{N1} & \ldots & vp_{NM} \\
  \end{array}
\right)$\\
where\\
$vp_{ij}=\{(1,if v_i  is assigned to p_j  @ 0,otherwise)\};1\leq i\leq N and 1\leq j\leq M$ \\


$vp_{ij}=\left(
\begin{array}{c}
1,if v_i  is assigned to p_j   \\
0,otherwise\\
  \end{array}
 \right) $

 For a given assignment $VP$, the $CPU$ utilization of $p_j$ can be calculated by\\
 $u_j=\left(
\sum_{i=1}^Nv_{cpu}^i*vp_{ij}
 \right)/p_cpu^j $ \\

 The energy consumption of $p_j$when turned on can be computed by\\

 $e_j=\left(
\begin{array}{c}
0,when \sum_{i=1}^Nvp_{ij}=0  \\
(e_{max}^j-e_{idle}^j)*\frac{u_j}{100}+e_{idle}^j,otherwise\\
  \end{array}
 \right) $ \\

 When $\sum_{i=1}^Nvp_{ij} =0$, it means that no virtual machine is assigned to
 $p_j$, so it can be turned off and cost no energy. \\

 The objective of the research is to find an assignment $VP$ that minimizes: \\
$\sum_{j=1}^Me_j$  \\

Subject to: \\

$\forall i,\sum_{j=1}^Mvp_{ij}=1$  \\


$\forall j \sum_{i=1}^Nv_{mem}^i*vp_{ij}\leq p_{mem}^j$ \\

$\forall j \sum_{i=1}^Nv_{cpu}^i*vp_{ij}\leq p_{cpu}^j$ \\

$\forall j \sum_{i=1}^Nv_{net}^i*vp_{ij}\leq p_{net}^j$ \\

Constraint (5) means one virtual machine can only be assigned to physical
machine; constraints (6,7,8) requires that the total assigned CPU, RAM, or
network resources cannot exceed the respective capacity on one host. Here we
implicitly assume that the overall resources on the Physical Machines (PMs) are
enough to accommodate the guest virtual machines, so we can ensure every guest
Virtual Machine (VM) can have a host to run on.

If we traverse all the combinations of VMs with PMs, there complexity is at
least $N^M$, so there will be no exact solution to the problem when $N$ and $M$
become a little larger. Therefore, we are going to deal with this problem via a
heuristic way.


\section{Algorithm description}

As the authors of SA suggests, there are basically four components in the
algorithm: 1) configuration of the system; 2) a random generator of the new
configuration;3) the objective function for the optimization problem; 4) a
schedule of the temperatures and length of times to evolve [12].In SAVMPA, the
four components are clearly defined to follow this methodology.

The assignment of virtual machines V to physical machines P is the
configuration. Making use of constraint that one VM can only assigned to one PM,
and for the convenience of implementation of the algorithm, we use an integer
array to represent the assignment rather than the matrix as in (0). \\
$VM=\{vm_0,vm_1,\ldots,vm_i,\ldots,vm_{n-1}\}$\\
The index of the array is the VM number, and the value indexed by the VM number
in the array is the PM number to which the VM is assigned.

At every evolution step, the configuration needs to be changed into a
``neighborhood'' state, so the configurations remain largely the unchanged
moving toward better configuration by searching in the neighborhood states. We
define a new neighborhood state as the new VM assignment array changed by
picking a small random number (from 1 to 3 in this implementation) of VMs and
changing their host machines to randomly selected PMs or swapping a pair of VM
assignment. These two types of rearrangement of VMs to create new configurations
reflect the basic operations of doing bin-packing to get a better result.

When a new configuration is feasible and have lower energy consumption, it will
be accepted as the new state. The criterion is the energy cost computed by the
total energy expressed the function (4). When the neighborhood searching gets
into a stalemate, we allow the new state cost more energy than the previous one,
so the algorithm would not be stuck in a local optimization. How much more
energy can be allowed is decided by the temperature. We define it as $e_l$:\\
$e_l=Max(e_{max}^j)*\frac{t_c}{t_0}$\\
Where $t_c$ is the current temperature and $t_0$ is the initial temperature.
The temperature starts from 1000 degree and reduces gradually to 0 by 5 degrees
each time, at which new temperature there are 10,000 times of evolutions.

Simulated Annealing VM placement Algorithm Description

\begin{algorithm}
\SetAlgoLined
\KwData{this text}
\KwResult{Simulated Annealing Virtual Machine Placement Algorithm }
%\caption{How to write algorithms}
\SetKwInOut{Input}{input}
\SetKwInOut{Output}{output}
\Input{VM requirements, PM capacities}
\Output{assignment of VMs to PMs}
	Generated Initial Assignment using FFD\; 
	$Temperature\leftarrow 1000$ \;
	\While{ Temperature $>$ 0 }{
	Set deviationEnergy = 0\;
	    \For{$i\leftarrow 1$ \KwTo 10000 }{
	        Create a random neighbor placement\;
	        \eIf{ the new placement is feasible \KwSty{and} incremental energy $<$
	        deviationEnergy\;}{ 
	               Set the placement as the new state\;
	               Set deviationEnergy = 0\;
	        }{
	             \eIf{ it is in stalemate }{
	                 Compute a deviation energy allowed by the current
	                 temperature\; } {}
	        }
	    }
	    Decrease $Temperature$\;
	}

\end{algorithm}


\section{Experiment and Evaluation}

To test the performance of SAVMP, we design the tests with different numbers of
VMs and PMs. In the tests, for the same number of VMs, PM sizes are varied with
capacity changed accordingly. Specifically, VM numbers are 20, 50, 100 or 200.
To show the size of containers, we introduce Capacity Index:\\
$CI=\frac{\sum_{j=1}^Mp_{cpu}^j / M)}{\sum_{i=1}^Nv_{cpu}^i / N}$ \\
If $CI$ is 1, that means the number of PMs needed is the same as the number of
VMs, and $CI=5$ means the PM number is only one fifth of that of VMs. In the
following experiment, for every number of VM, CI values are multiple, trying
each in the set$ [1, 2, 3, 4, 5, 10]$.
The VM CPU requirement is generated randomly from the value 0 to 2000. This is
to simulate the resource usage of the VMs at the data center at a sampling time.
The PM sizes vary from 1000 to 3000 multiplied by the CI (capacity index),
simulating the heterogeneous host environment.

In one group of tests, we only consider CPU requirements and in another we add
memory requirements to test the performance of the SAVMP on one dimensional and
two dimensional packing respectively. The requirement of memory is generated in
the same way as CPU, but the two requirements are most probably not the same
value for the same VM because of the random nature.
Each PM’s maximum energy $e_j$ is given by the equation:\\
$e_j=(1-\log (p_{cpu}^j/1000)*0.4)*E*(p_{cpu}^j/1000)$

Where E is a const, set to 100 Watt in our test, representing the base energy
which is consumed by the smallest PM when $p_{cpu}^j=1000$. When the CPU
capability is 10 times of the smallest PM, the energy cost is only 6 times as
much. This equation reflects the energy saving method by adopting larger
physical machines.

For each physical server, the energy of idle sqtatus is set to 70% of the
% maximum power, namely,\\
$e_{idle}^j= e_{max}^j*0.7$.


As to the initial placement, our proposed SAVMP does not require a specified
state. In the following tests, we choose to start from the placement results
generated by FFD, although SAVMP can achieve a comparable result from a random
feasible placement with the cost of more time according to our preliminary
tests.

For every configuration of VMs and PMs, we run ten times and average the energy
saving compared to FFD and the time to reach the best placement among all the
evolutions in order to have a statistical view of performance of the SAVMP
because it does not generate the exact same result every time as FFD does for
the same problem. All the tests are run on a laptop with a 2.2G Hz dual core
processor.

Table1 : Simulation test results of SAVMP on diqfferent numbers of VMs and
varied Capacity Indexes with only CPU constraints

$\begin{tabular}{c|c|c|c|c|c}
VM  & CI & Energy  & Energy   & Energy & Time (s)\\
No & & by FFD & by SA &  Saved & \\
  & & (Watt) & (Watt) & & \\
\hline
20 & 1 & 1691.8 & 1536.3 & 9.18\% & 3.18\\
20 & 2 & 1445.9 & 1351.1 & 6.56\% & 0.62\\
20 & 3 & 1491.2 & 1428.3 & 4.22\% & 0.35\\
20 & 4 & 1341.8 & 1271.8 & 5.21\% & 1.16\\
20 & 5 & 1374.9 & 1339.3 & 2.59\% & 0.60\\
20 & 10 & 1219.1 & 1219.0 & 0.01\% & 0.94\\
\hline
50 & 1 & 4283.5 & 4040.8 & 5.40\% & 28.42\\
50 & 2 & 3615.0 & 3507.1 & 2.99\% & 2.21\\
50 & 3 & 3297.5 & 3188.0 & 3.33\% & 6.25\\
50 & 4 & 3005.0 & 3004.7 & 0.01\% & 0.42\\
50 & 5 & 2839.3 & 2839.1 & 0.01\% & 0.54\\
50 & 10 & 2576.9 & 2576.7 & 0.01\% & 0.98\\
\hline
100 & 1 & 9399.5 & 8654.4 & 8.12\% & 282.22\\
100 & 2 & 7444.7 & 7327.9 & 1.10\% & 49.02\\
100 & 3 & 6880.4 & 6728.9 & 2.20\% & 8.45\\
100 & 4 & 6471.1 & 6112.6 & 5.54\% & 0.61\\
100 & 5 & 5774.0 & 5773.6 & 0.01\% & 20.11\\
100 & 10 & 4510.5 & 4510.3 & 0.00\% & 1.01\\
\hline
200 & 1 & 18234.2 & 16837.6 & 7.99\% & 1830.65\\
200 & 2 & 14516.2 & 14269.8 & 1.94\% & 1952.39\\
200 & 3 & 13001.2 & 12853.7 & 0.57\% & 170.54\\
200 & 4 & 11846.3 & 11845.3 & 0.01\% & 51.37\\
200 & 5 & 11447.5 & 11269.3 & 1.68\% & 23.09\\
200 & 10 & 8941.6 & 8941.3 & 0.00\% & 27.71
\end{tabular}$

From Table1, we can see SAVMP is able to generate better results than FFD for
most of the situations and at least no worse results than FFD for all the tests.
This may partially attribute to starting from the FFD placement, but this also
shows one of its good characteristics of starting from any feasible placement
and improving it little by little.

However, the improvement by SAVMP varies with the VM numbers and CI. The larger
is the VM number or CI, the less is the improvement. This can be attributed to 2
possible reasons. One is that with the larger number of VMs, the search space
becomes larger; another one is that the performance of FFD varies on the
different problems. In Table-2, the placement results reveal that FFD can have a
very good result when VM number is large and CI is also big.

Almost all the tests can achieve the best results among all the iterations in
less than 100 seconds. For the tests with exceptionally long time to reach the
``best'' results, we find that they can achieve very good results compared to
the ``best'' saving value within 100 seconds after looking into the evolution
history records.

Table 2:  VM placement results of FFD on 200 VMs (capacity index = 10)

$\begin{tabular}{c|c|c|c|c|c|c}
PM & Size & CPU\% & & PM & Size & CPU\% \\
\hline
0 & 30000 & 99.67\% & & 10 & 20000 & 0.00\%\\
1 & 30000 & 99.67\% & & 11 & 20000 & 0.00\%\\
2 & 27000 & 99.63\% & & 12 & 18000 & 0.00\%\\
3 & 27000 & 99.63\% & & 13 & 18000 & 0.00\%\\
4 & 25000 & 99.60\% & & 14 & 15000 & 0.00\%\\
5 & 25000 & 99.60\% & & 15 & 15000 & 0.00\%\\
6 & 24000 & 99.58\% & & 16 & 12000 & 0.00\%\\
7 & 24000 & 48.33\% & & 17 & 12000 & 0.00\%\\
8 & 23000 & 0.00\% & & 18 & 10000 & 0.00\%\\
9 & 23000 & 0.00\% & & 19 & 10000 & 0.00\%\\
\end{tabular}$


In Table-3, which contains the results by FFD and SAVMP with CPU and memory
constraints, the columns are the same as in Table-1 except that there is an
extra ``Energy saved by Random Search'' column in Table-3, which lists the
energy saving effects achieved by a random search method. This method picks a PM
randomly for each VM. If the chosen PM does not satisfy the constraints, it will
try another randomly chosen PM until it finds a suitable one or give up
searching and return no feasible solution after the maximum attempts have been
used. The same process repeats until all the VMs have their assigned PMs.

The Random Search performs comparable to SAVMP when there are only 20 VMs.
However, when the VM number increases, the energy saving improvement becomes
worse. This is especially obvious when the number is greater than 50. The Random
Search can only improve by a small percentage to FFD when CI is 5 or 10. In all
tests but one test of which the VM number is 20 and CI is 5, the Random Search
performs worse than SAVMP.

We add the random methodology to prove that the SAVMP performs better than FFD
not only because it involves random searching method to explore more assignment
combinations than FFD, but also because the Simulation Annealing approach
provides a guided search more than random attempts.

The improvement effect by SAVMP for two constraints is more significant than
that for only CPU constraint. When the VM number becomes larger, the improvement
tends to be less because the searching space grows bigger. The relationship of
the improvement and CI is not straightforward. It has two contradictory effects
on the improvement by SAVMP. For one thing, the larger the CI is, the smaller
the searching space becomes. The smaller searching space favors SAVMP but FFD
also performs better than when CI is smaller.

From the table, we can find another benefit of using SAVMP in the test of which
the VM number is 20 and CI is 10. FFD or Random Search cannot find a feasible
solution for this test, but SAVMP can find the assignment successfully.

Table3 : Simulation test results of SAVMP on different numbers of VMs and varied
Capacity Indexes with CPU and Memory constraints

$\begin{tabular}{c|c|c|c|c|c|c}
 VM  & CI & Energy  & Energy   & Time  & Energy &Energy\\
No & & by FFD & by SA & (s)  &Saved &saved by\\
  & & (Watt) & (Watt) & & &RS\\
\hline 
20 & 1 & 2090.7 & 1774.1 & 3.20 & 15.79\% & 9.23\%\\
20 & 2 & 1682.7 & 1445.5 & 1.54 & 14.13\% & 8.35\%\\
20 & 3 & 1688.9 & 1429.2 & 1.24 & 15.37\% & 13.91\%\\
20 & 4 & 1800.3 & 1341.6 & 0.52 & 25.48\% & 24.45\%\\
20 & 5 & 1627.7 & 1339.3 & 1.32 & 16.84\% & 17.40\%\\
20 & 10 & N/A* & 1219.0 & 0.94 & N/A* & N/A*\\
\hline 
50 & 1 & 4834.7 & 4310.9 & 28.92 & 10.19\% & 0.17\%\\
50 & 2 & 4256.4 & 3616.8 & 7.45 & 15.15\% & 5.81\%\\
50 & 3 & 3856.6 & 3298.8 & 1.73 & 14.47\% & 7.02\%\\
50 & 4 & 3365.2 & 3006.5 & 0.72 & 10.67\% & 3.82\%\\
50 & 5 & 3304.5 & 2840.8 & 0.64 & 14.04\% & 11.43\%\\
50 & 10 & 3057.3 & 2578.2 & 1.09 & 15.68\% & 15.65\%\\
\hline 
100 & 1 & 9404.8 & 8871.7 & 131.53 & 5.41\% & 0\%\\
100 & 2 & 8159.5 & 7448.1 & 44.51 & 8.78\% & 0\%\\
100 & 3 & 7494.9 & 6813.5 & 25.86 & 9.08\% & 0.57\%\\
100 & 4 & 6835.5 & 6116.3 & 7.78 & 10.53\% & 0\%\\
100 & 5 & 6704.9 & 5777.0 & 2.04 & 13.85\% & 6.37\%\\
100 & 10 & 5247.0 & 4513.0 & 1.85 & 14.00\% & 7.41\%\\
\hline 
200 & 1 & 18373.2 & 17379.2 & 1250.81 & 5.46\% & 0\%\\
200 & 2 & 15944.7 & 14514.2 & 208.13 & 8.29\% & 0\%\\
200 & 3 & 14606.7 & 12999.5 & 215.01 & 10.14\% & 0\%\\
200 & 4 & 13442.6 & 12253.2 & 22.45 & 8.85\% & 0\%\\
200 & 5 & 12375.4 & 11446.7 & 17.41 & 7.50\% & 1.66\%\\
200 & 10 & 9675.6 & 8941.3 & 23.50 & 7.59\% & 5.61\% 
\end{tabular}$

*Note: For VM number=2 and CI=10 with CPU and memory constraints, FFD cannot
provide a feasible solution, so the ``Energy Saved'' value is not applicable.

\newpage
It is very hard to determine the minimum energy usage among all the possible
assignment, but we can use the resource utilization on the PMs resulted from the
VM assignment to examine and compare the energy saving effect by different
algorithms. In Table-4 and Table-5, the utilization of CPU and memory has been
listed for each PM for the test of 200 VMs with CI set to 10 applying FFD and
SAVMP respectively. It is clear that SAVMP achieve higher resource utilization
than FFD, and leaves very small room for further improvement.

Table 4 - PM resource utilization of Placement by FFD on 200 VMs with capacity index = 10
\begin{center}
$\begin{tabular}{c|c|c|c}
PM & Size & CPU & MEM \\
 &  & \% & \% \\
\hline
0 & 30000 & 99.83\% & 54.00\% \\
1 & 30000 & 99.83\% & 58.67\% \\
2 & 27000 & 99.81\% & 75.93\%\\
3 & 27000 & 99.81\% & 57.04\% \\
4 & 25000 & 99.80\% & 83.20\% \\
5 & 25000 & 99.80\% & 94.00\%\\
6 & 24000 & 80.00\% & 99.58\% \\
7 & 24000 & 51.25\% & 99.58\% \\
8 & 23000 & 18.04\% & 99.57\% \\
9 & 23000 & 0.43\% & 13.91\% \\
10 & 20000 & 0.00\% & 0.00\%\\
11 & 20000 & 0.00\% & 0.00\%\\
12 & 18000 & 0.00\% & 0.00\%\\
13 & 18000 & 0.00\% & 0.00\%\\
14 & 15000 & 0.00\% & 0.00\%\\
15 & 15000 & 0.00\% & 0.00\%\\
16 & 12000 & 0.00\% & 0.00\%\\
17 & 12000 & 0.00\% & 0.00\%\\
18 & 10000 & 0.00\% & 0.00\%\\
19 & 10000 & 0.00\% & 0.00\%
\end{tabular}$

\end{center}

  
                 
%\begin{center}
Table 5 - Annealing results of 200 VMs\\ 
(capacity index = 10, CPU and MEM constraints)
$\begin{tabular}{c|c|c|c}

PM & Size & CPU & MEM \\
 &  & \% & \%  \\
\hline
0 & 30000 & 100.00\% & 79.00\%\\
1 & 30000 & 100.00\% & 96.67\%\\
2 & 27000 & 100.00\% & 82.96\%\\
3 & 27000 & 100.00\% & 78.15\%\\
4 & 25000 & 99.80\% & 99.20\%\\
5 & 25000 & 99.80\% & 98.40\% \\
6 & 24000 & 74.79\% & 85.00\% \\
7 & 24000 & 73.33\% & 91.25\% \\
8 & 23000 & 0.00\% & 0.00\% \\
9 & 23000 & 0.00\% & 0.00\%\\
11 & 20000 & 0.00\% & 0.00\%\\
10 & 20000 & 0.00\% & 0.00\%\\
12 & 18000 & 0.00\% & 0.00\%\\
13 & 18000 & 0.00\% & 0.00\%\\
14 & 15000 & 0.00\% & 0.00\%\\
15 & 15000 & 0.00\% & 0.00\%\\
16 & 12000 & 0.00\% & 0.00\%\\
17 & 12000 & 0.00\% & 0.00\%\\
18 & 10000 & 0.00\% & 0.00\%\\
19 & 10000 & 0.00\% & 0.00\%
\end{tabular}$

%\end{center}

\section{Conclusion and Future work}

We propose a VM placement algorithm based on Simulated Annealing methodology to
improve energy efficiency in data centers. The First Fit Decreasing algorithm
for bin-packing is selected for comparison in evaluating the performance of the
algorithm. Extensive experiment has been done with different problem
configurations. The number of VMs varies from 20 to 200, the sizes the PMs are
averagely one to 10 times of those of VMs, and resource constraints are CPU only
- one dimensional, and CPU combined with memory a�� two dimensions.

Experiment results show that it can generate better VM assignment than FFD,
especially in the two dimensional constraint problems with 0-25\% more energy
saving than FFD in a reasonable time limit. How much improvement it can make
from the FFD depends on a number of factors: the larger the problem size, the
less the improvement; the better is FFD fit for the target problem, the less the
improvement; two dimensional problems leave more room for improvement.

Besides the more effective optimization result, the SAVMP has another benefit:
it generates multiple placements that have better optimization result than FFD.
This makes the algorithm potential to be used in the dynamic VM allocation to
minimize the migration cost while saving energy. In our future work we will
attempt using SAVMPA in live VM migration solution and compare its effectiveness
with existing solutions.


%\newpage
\section{Reference}

\end{document}